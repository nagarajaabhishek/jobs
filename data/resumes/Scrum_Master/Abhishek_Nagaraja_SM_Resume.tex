%----------------------------------------------------------------------------------------
%   CLASS DEFINITIONS (EMBEDDED RESUME.CLS)
%----------------------------------------------------------------------------------------
\documentclass[9pt,letterpaper]{article} % Font size and paper type

\usepackage[parfill]{parskip} % Remove paragraph indentation
\usepackage{array} % Required for boldface (\bf and \bfseries) tabular columns
\usepackage{ifthen} % Required for ifthenelse statements
\usepackage{charter}
\usepackage{enumitem}
\usepackage{xcolor} 
\usepackage[colorlinks=true, urlcolor=blue]{hyperref}
\usepackage[left=0.5in,top=0.25in,right=0.4in,bottom=0.2in, columnsep=0.5cm]{geometry} 
\input{glyphtounicode}
\usepackage[none]{hyphenat}
\usepackage{multicol}

\pdfgentounicode=1
\pagestyle{empty} % Suppress page numbers

\makeatletter % Begin internal command redefinitions

%----------------------------------------------------------------------------------------
%	HEADINGS COMMANDS: Commands for printing name and address
%----------------------------------------------------------------------------------------

\def \name#1{\def\@name{#1}} % Defines the \name command to set name
\def \@name {} % Sets \@name to empty by default

\def \addressSep {$\diamond$} % Set default address separator to a diamond

% One, two or three address lines can be specified 
\let \@addressone \relax
\let \@addresstwo \relax
\let \@addressthree \relax

% \address command can be used to set the first, second, and third address (last 2 optional)
\def \address #1{
  \@ifundefined{@addresstwo}{
    \def \@addresstwo {#1}
  }{
  \@ifundefined{@addressthree}{
  \def \@addressthree {#1}
  }{
     \def \@addressone {#1}
  }}
}

% \printaddress is used to style an address line (given as input)
\def \printaddress #1{
  \begingroup
    \def \\ {\addressSep\ }
    \centerline{#1}
  \endgroup
  \par
}

% \printname is used to print the name as a page header
\def \printname {
  \begingroup
    \hfil{{\namesize\bf \@name}}\hfil
    \nameskip\break
  \endgroup
}

%----------------------------------------------------------------------------------------
%	PRINT THE HEADING LINES
%----------------------------------------------------------------------------------------

\let\ori@document=\document
\renewcommand{\document}{
  \ori@document  % Begin document
  \printname % Print the name specified with \name
  \@ifundefined{@addressone}{}{ % Print the first address if specified
    \printaddress{\@addressone}}
  \@ifundefined{@addresstwo}{}{ % Print the second address if specified
    \printaddress{\@addresstwo}}
     \@ifundefined{@addressthree}{}{ % Print the third address if specified
    \printaddress{\@addressthree}}
}

%----------------------------------------------------------------------------------------
%	SECTION FORMATTING
%----------------------------------------------------------------------------------------

% Defines the rSection environment for the large sections within the CV
\newenvironment{rSection}[1]{ % 1 input argument - section name
  \sectionskip
 \textbf{\Large #1} % Section title
  \sectionlineskip
  \hrule % Horizontal line
  \begin{list}{}{ % List for each individual item in the section
    \setlength{\leftmargin}{-0em} % Margin within the section
  }
  \item[]
}{
  \end{list}
}

%----------------------------------------------------------------------------------------
%	WORK EXPERIENCE FORMATTING
%----------------------------------------------------------------------------------------

\newenvironment{rSubsection}[4]{ % 4 input arguments - company name, year(s) employed, job title and location
 {\bf #1} \hfill {#2} % Bold company name and date on the right
 \ifthenelse{\equal{#3}{}}{}{ % If the third argument is not specified, don't print the job title and location line
  \\
  {\em #3} \hfill {\em #4} % Italic job title and location
  }\smallskip
  \begin{list}{$\cdot$}{\leftmargin=0em} % \cdot used for bullets, no indentation
   \itemsep -0.5em \vspace{-0.5em} % Compress items in list together for aesthetics
  }{
  \end{list}
  \vspace{0em} % Some space after the list of bullet points
}

% The below commands define the whitespace after certain things in the document - they can be \smallskip, \medskip or \bigskip
\def\namesize{\huge} % Size of the name at the top of the document
\def\addressskip{} % The space between the two address (or phone/email) lines
\def\sectionlineskip{\smallskip} % The space above the horizontal line for each section 
\def\nameskip{\vspace{-0.5em}} % The space after your name at the top
\def\sectionskip{} % The space after the heading section

\makeatother % End internal command redefinitions

%----------------------------------------------------------------------------------------
%   DOCUMENT START
%----------------------------------------------------------------------------------------
\name{Abhishek Nagaraja, (CSPO) \vspace{.2em} \hrule}

\address{Arlington, Texas, USA \textbullet\ work.abhishekn@gmail.com \textbullet\ {\href{https://www.abhishekn.in}{www.abhishekn.in}} \textbullet\ {\href{https://linkedin.com/in/abhisheknagaraja}{linkedin.com/in/abhisheknagaraja}}}

\begin{document}

%----------------------------------------------------------------------------------------
%   SUMMARY
%----------------------------------------------------------------------------------------
Certified \textbf{Scrum Master (CSPO)} and \textbf{Agile Practitioner} with a strong foundation in \textbf{Servant Leadership}, \textbf{Stakeholder Networking}, and \textbf{Process Improvement}. Proven ability to facilitate \textbf{Scrum Ceremonies}, remove \textbf{Impediments}, and improve \textbf{Team Velocity} by fostering a culture of collaboration. Skilled in leveraging \textbf{Product Storytelling} to align teams and \textbf{Generative AI} for backlog refinement.

\begin{rSection}{Education}
\begin{rSubsection}{University of Texas at Arlington}{Arlington, Texas, USA}{\textbf{Master of Science in Computer Science}}{Aug 2023 - May 2024}
\item \textbf{Master of Science in Engineering Management} \hfill May 2024 - Current
\end{rSubsection}

\begin{rSubsection}{Jawaharlal Nehru Technological University}{Hyderabad, India}{\textbf{Bachelor of Technology in Computer Science and Engineering}}{Aug 2019 - July 2023}
\item[] \vspace{-1em}
\end{rSubsection}
\end{rSection}


%----------------------------------------------------------------------------------------
%   WORK EXPERIENCE SECTION
%----------------------------------------------------------------------------------------
\begin{rSection}{Work Experience}

\begin{rSubsection}
{Program Manager Intern, University of Texas at Arlington - Texas, USA}{October 2023 - October 2025}{}{}
\item[] \vspace{-0.5em}
\begin{itemize}[left=-1.5em, labelsep=.1cm, labelwidth=.2cm, itemsep=0.0em]
\item \textbf{Facilitated collaborative Sprint Planning sessions} achieving \textbf{90\% sprint commitment attainment} for 'MavMarket' (scaling to 5,000+ users), utilizing \textbf{Jira} for capacity planning and ensuring all backlog items met the \textbf{Definition of Ready (DoR)}.
\item \textbf{Orchestrated highly focused Daily Stand-ups} across 7 workstreams to drive \textbf{seamless cross-functional alignment}, leveraging strategic network building to resolve inter-departmental impediments in real-time and reduced dependency blockers.
\item \textbf{Engineered an AI-powered SOP} that \textbf{reduced cycle time by 40\%}, by automating technical documentation workflows and proactively identifying process bottlenecks to streamline impediments.
\item \textbf{Conducted data-driven Sprint Retrospectives} for 300+ stakeholders, resulting in a \textbf{35\% increase in team satisfaction (NPS)}, by implementing actionable feedback loops and fostering a culture of continuous improvement.
\item \textbf{Coached the development team} on \textbf{Agile values and principles}, ensuring strict adherence to Scrum ceremonies and significantly improving overall sprint completion rates through servant leadership.
\end{itemize}
\end{rSubsection}

\begin{rSubsection}
{Business Analyst Intern, NFC Solutions Private Limited - Hyderabad, India}{April 2022 - January 2023}{}{}
\item[] \vspace{-0.5em}
\begin{itemize}[left=-1.5em, labelsep=.1cm, labelwidth=.2cm, itemsep=0.0em]
\item \textbf{Served as Junior Scrum Master} for 4 high-priority technical projects, driving a \textbf{20\% improvement in team velocity} by removing non-technical impediments and protecting the development team from scope creep during active sprints.
\item \textbf{Accelerated backlog readiness} and reduced refinement cycle time, by \textbf{leveraging Large Language Models (LLMs)} to draft comprehensive Acceptance Criteria and assisting the Product Owner in prioritizing high-value user stories according to \textbf{WSJF (Weighted Shortest Job First)}.
\item \textbf{Managed the sprint board in Jira} ensuring \textbf{100\% visibility} of work, by maintaining accurate burndown charts and tracking task progress to facilitate transparent communication with stakeholders.
\item \textbf{Coordinated interactive Sprint Reviews} and demos to facilitate user feedback sessions, validating feature releases against business requirements and ensuring immediate value delivery.
\item \textbf{Collaborated with QA and Dev teams} to resolve blockers, ensuring system integration met high-quality standards with \textbf{zero production defects} through rigorous definition of done (DoD) enforcement.
\end{itemize}
\end{rSubsection}

\begin{rSubsection}{Founder and Community Lead, e-DAM Community - Hyderabad, India}{March 2021 - January 2025}{}{}
\item[] \vspace{-0.5em}
\begin{itemize}[left=-1.5em, labelsep=.1cm, labelwidth=.2cm, itemsep=0.0em]
\item \textbf{Scaled the technical community} to \textbf{5,000+ members} by \textbf{adopting an Agile approach} to content strategy and \textbf{building a robust network} of student leaders and mentors.
\item \textbf{Facilitated community retrospectives} as measured by \textbf{high-value attendee feedback}, by using \textbf{narrative storytelling} and active engagement to drive continuous improvement iterations.
\end{itemize}
\end{rSubsection}

\end{rSection}

%----------------------------------------------------------------------------------------
%   PROJECTS
%----------------------------------------------------------------------------------------
\begin{rSection}{Projects}

\begin{rSubsection}{Thara: Multi-Agent AI Ecosystem}{}{}{}
\item[] \vspace{-0.5em}
\begin{itemize}[left=-1.5em, labelsep=.1cm, labelwidth=.2cm, itemsep=0.0em]
\item \textbf{Managed the Scrum Board} for \textbf{11 system components} achieving efficient sprint execution, by ensuring clear \textbf{Acceptance Criteria} and visibility for the development team on complex architectural dependencies.
\item \textbf{Secured 99.9\% system uptime reliability} by \textbf{negotiating Technical Debt remediation} during sprint planning, ensuring critical architectural fixes were prioritized in the \textbf{Product Backlog} alongside new feature development.
\end{itemize}
\end{rSubsection}

\begin{rSubsection}{Google Maps: Product Teardown}{}{}{}
\item[] \vspace{-0.5em}
\begin{itemize}[left=-1.5em, labelsep=.1cm, labelwidth=.2cm, itemsep=0.0em]
\item \textbf{Validated roadmap prioritization} through \textbf{42 user interviews}, conducting qualitative research to advocate for high-impact features and help the Product Owner align the backlog with user needs.
\item \textbf{Secured stakeholder buy-in} through \textbf{data storytelling}, by \textbf{assisting in Sprint Reviews} and demonstrating value increments to key leadership stakeholders.
\end{itemize}
\end{rSubsection}

\end{rSection}

%----------------------------------------------------------------------------------------
%   SKILLS
%----------------------------------------------------------------------------------------
\begin{rSection}{Skills and Tools}
\item[] \vspace{-0.5em}
\begin{itemize}[left=-1.5em, labelsep=.1cm, labelwidth=.2cm, itemsep=0.0em]
    \item \textbf{Agile Frameworks:} Scrum, Kanban, Lean, XP, SAFe (Concepts).
    \item \textbf{Scrum Mastery:} Facilitation, Servant Leadership, Conflict Resolution, Impediment Removal, Backlog Refinement, Retrospectives, Velocity Tracking, Burndown Charts.
    \item \textbf{Soft Skills:} \textbf{Product Storytelling}, \textbf{Network Building}, Stakeholder Management, Communication.
    \item \textbf{Tools:} Jira, \textbf{Linear}, Confluence, Trello, Asana, Mural, Miro, Slack, Zoom, \textbf{Microsoft Office (Word, Excel, PowerPoint)}, \textbf{Google Workspace (Docs, Sheets)}.
    \item \textbf{AI \& Tech:} \textbf{Generative AI (ChatGPT/Gemini)} for Documentation, \textbf{Predictive Analytics}, SDLC, CI/CD, \textbf{Clerk (Auth)}, \textbf{NeonDB}.
\end{itemize}
\end{rSection}

\begin{rSection}{Certification}
\item[] \vspace{-1em}
\begin{itemize}[left=-1.5em, labelsep=.1cm, labelwidth=.2cm, itemsep=0.0em]
\item Certified Scrum Product Owner (CSPO)
\item Google: Project Management Professional Certificate
\item Google: Agile Project Management 
\end{itemize}
\end{rSection}


\end{document}
