%----------------------------------------------------------------------------------------
%   CLASS DEFINITIONS (EMBEDDED RESUME.CLS)
%----------------------------------------------------------------------------------------
\documentclass[9pt,letterpaper]{article} % Font size and paper type

\usepackage[parskip]{parskip} % Remove paragraph indentation
\usepackage{array} % Required for boldface (\bf and \bfseries) tabular columns
\usepackage{ifthen} % Required for ifthenelse statements
\usepackage{charter}
\usepackage{enumitem}
\usepackage{xcolor}
\usepackage[colorlinks=true, urlcolor=blue]{hyperref}
\usepackage[left=0.5in,top=0.25in,right=0.4in,bottom=0.2in, columnsep=0.5cm]{geometry}
\input{glyphtounicode}
\usepackage[none]{hyphenat}
\usepackage{multicol}

\pdfgentounicode=1
\pagestyle{empty} % Suppress page numbers

\makeatletter % Begin internal command redefinitions

%----------------------------------------------------------------------------------------
%	HEADINGS COMMANDS: Commands for printing name and address
%----------------------------------------------------------------------------------------

\def \name#1{\def\@name{#1}} % Defines the \name command to set name
\def \@name {} % Sets \@name to empty by default

\def \addressSep {$\diamond$} % Set default address separator to a diamond

% One, two or three address lines can be specified
\let \@addressone \relax
\let \@addresstwo \relax
\let \@addressthree \relax

% \address command can be used to set the first, second, and third address (last 2 optional)
\def \address #1{
  \@ifundefined{@addresstwo}{
    \def \@addresstwo {#1}
  }{
  \@ifundefined{@addressthree}{
  \def \@addressthree {#1}
  }{
     \def \@addressone {#1}
  }}
}

% \printaddress is used to style an address line (given as input)
\def \printaddress #1{
  \begingroup
    \def \\ {\addressSep\ }
    \centerline{#1}
  \endgroup
  \par
}

% \printname is used to print the name as a page header
\def \printname {
  \begingroup
    \hfil{{\namesize\bf \@name}}\hfil
    \nameskip\break
  \endgroup
}

%----------------------------------------------------------------------------------------
%	PRINT THE HEADING LINES
%----------------------------------------------------------------------------------------

\let\ori@document=\document
\renewcommand{\document}{
  \ori@document  % Begin document
  \printname % Print the name specified with \name
  \@ifundefined{@addressone}{}{ % Print the first address if specified
    \printaddress{\@addressone}}
  \@ifundefined{@addresstwo}{}{ % Print the second address if specified
    \printaddress{\@addresstwo}}
     \@ifundefined{@addressthree}{}{ % Print the third address if specified
    \printaddress{\@addressthree}}
}

%----------------------------------------------------------------------------------------
%	SECTION FORMATTING
%----------------------------------------------------------------------------------------

% Defines the rSection environment for the large sections within the CV
\newenvironment{rSection}[1]{ % 1 input argument - section name
  \sectionskip
 \textbf{\Large #1} % Section title
  \sectionlineskip
  \hrule % Horizontal line
  \begin{list}{}{ % List for each individual item in the section
    \setlength{\leftmargin}{-0em} % Margin within the section
  }
  \item[]
}{
  \end{list}
}

%----------------------------------------------------------------------------------------
%	WORK EXPERIENCE FORMATTING
%----------------------------------------------------------------------------------------

\newenvironment{rSubsection}[4]{ % 4 input arguments - company name, year(s) employed, job title and location
 {\bf #1} \hfill {#2} % Bold company name and date on the right
 \ifthenelse{\equal{#3}{}}{}{ % If the third argument is not specified, don't print the job title and location line
  \\
  {\em #3} \hfill {\em #4} % Italic job title and location
  }\smallskip
  \begin{list}{$\cdot$}{\leftmargin=0em} % \cdot used for bullets, no indentation
   \itemsep -0.5em \vspace{-0.5em} % Compress items in list together for aesthetics
  }{
  \end{list}
  \vspace{0em} % Some space after the list of bullet points
}

% The below commands define the whitespace after certain things in the document - they can be \smallskip, \medskip or \bigskip
\def\namesize{\huge} % Size of the name at the top of the document
\def\addressskip{} % The space between the two address (or phone/email) lines
\def\sectionlineskip{\smallskip} % The space above the horizontal line for each section 
\def\nameskip{\vspace{-0.5em}} % The space after your name at the top
\def\sectionskip{} % The space after the heading section

\makeatother % End internal command redefinitions

%----------------------------------------------------------------------------------------
%   DOCUMENT START
%----------------------------------------------------------------------------------------
\name{Abhishek Nagaraja \vspace{.2em} \hrule}

\address{Arlington, Texas, USA - 76013 \textbullet\ work.abhishekn@gmail.com \textbullet\ \href{https://www.linkedin.com/in/abhisheknagaraja/}{linkedin.com/in/abhisheknagaraja} \textbullet\ \href{https://www.abhishekn.in/}{www.abhishekn.in}}

\begin{document}

%----------------------------------------------------------------------------------------
%   SUMMARY
%----------------------------------------------------------------------------------------
Product \& Growth Professional specializing in \textbf{GTM Strategy}, \textbf{Community Building}, and \textbf{Technical Product Marketing}. Expert in orchestrating two-sided marketplace launches and bridging the gap between engineering innovation and commercial adoption. Proven ability to drive user acquisition, optimize revenue operations, and build thriving technical communities.

\begin{rSection}{Education}
\begin{rSubsection}{University of Texas at Arlington}{Arlington, Texas, USA}{\textbf{Master of Science in Computer Science}}{Aug 2023 - May 2024}
\item \textbf{Master of Science in Engineering Management} \hfill May 2024 - Current
\end{rSubsection}

\begin{rSubsection}{Jawaharlal Nehru Technological University}{Hyderabad, India}{\textbf{Bachelor of Technology in Computer Science and Engineering}}{Aug 2019 - July 2023}
\item[] \vspace{-1em}
\end{rSubsection}
\end{rSection}

%----------------------------------------------------------------------------------------
%   WORK EXPERIENCE SECTION
%----------------------------------------------------------------------------------------
\begin{rSection}{Work Experience}

\begin{rSubsection}
{Program Manager Intern, University of Texas at Arlington - Texas, USA}{October 2023 - Present}{}{}
\item[] \vspace{-0.5em}
\begin{itemize}[left=-1.5em, labelsep=.1cm, labelwidth=.2cm, itemsep=0.0em]
\item \textbf{Orchestrated the Go-to-Market (GTM) strategy} for 'MavMarket', scaling a two-sided marketplace to \textbf{5,000+ attendees and 200+ vendors}, by identifying key user personas and executing a multi-channel acquisition campaign.
\item \textbf{Generated \$120,000 in collective revenue} through data-driven vendor recruitment and strategic pricing optimization, analyzing historical transaction data to identify high-value segments and maximize vendor retention.
\item \textbf{Drove User Acquisition \& Engagement} by designing feedback loops for 300+ stakeholders, increasing vendor NPS by \textbf{35\%}, by implementing a real-time sentiment analysis dashboard that informed rapid product iteration cycles.
\item \textbf{Optimized Revenue Operations} by architecting a \textbf{50+ page, AI-powered SOP}, reducing event planning overhead by \textbf{40\%} and streamlining cross-functional workflows to ensure consistent execution across all marketplace touchpoints.
\item \textbf{Led Community Growth} for the Entrepreneurship Club, growing active membership to \textbf{100+} through targeted outreach, by developing a "Founder First" content strategy that resonated with the university's aspiring builder demographic.
\item \textbf{Facilitated Product Storytelling} workshops to align technical workstreams with market-facing value propositions, ensuring that engineering features were translated into compelling user benefits for external stakeholders.
\end{itemize}
\end{rSubsection}

\begin{rSubsection}{Founder and Community Lead, e-DAM Community - Hyderabad, India}{March 2021 - January 2025}{}{}
\item[] \vspace{-0.5em}
\begin{itemize}[left=-1.5em, labelsep=.1cm, labelwidth=.2cm, itemsep=0.0em]
\item \textbf{Engineered a Community-Led Growth} engine, scaling the Hyderabad Technical Community to \textbf{5,000+ students}, by creating a viral loop of peer-to-peer referral incentives and high-value exclusive content drops.
\item \textbf{Delivered 200+ technical sessions} and workshops to drive brand awareness and facilitate peer-to-peer learning, establishing e-DAM as the premier destination for upskilling in the regional student developer ecosystem.
\item \textbf{Executed Strategic Partnerships} with 7+ institutions, fostering a talent pipeline and generating \textbf{50+ internships}, by negotiating mutually beneficial MOUs that aligned corporate hiring needs with student skill development.
\item \textbf{Managed Brand \& Event Strategy} for "Social Media Summit 2023", securing \textbf{Rs. 4,00,000+ in sponsorship} and 500+ attendees, by crafting a compelling sponsorship deck that highlighted audience demographics and engagement metrics.
\item \textbf{Launched the 'e-DAM Spotlight' podcast} to amplify founder stories, driving engagement across the startup ecosystem, by curating a diverse guest list of regional innovators to inspire the next generation of builders.
\item \textbf{Built a robust network} of \textbf{10+ influencers and industry mentors} to support community initiatives and growth, leveraging their social reach to amplify campaign messaging and attract high-quality speakers.
\end{itemize}
\end{rSubsection}

\begin{rSubsection}
{Business Analyst Intern, NFC Solutions Private Limited - Hyderabad, India}{April 2022 - January 2023}{}{}
\item[] \vspace{-0.5em}
\begin{itemize}[left=-1.5em, labelsep=.1cm, labelwidth=.2cm, itemsep=0.0em]
\item \textbf{Managed Client Delivery} for 4 high-priority projects, authoring detailed \textbf{PRDs} to align engineering with business goals, ensuring that all functional requirements were mapped directly to client success metrics.
\item \textbf{Enhanced Operational Efficiency} by \textbf{20\%} as Scrum Master, implementing Lean-Agile workflows for faster time-to-market, by removing process bottlenecks and optimizing sprint ceremonies for maximum team velocity.
\item \textbf{Optimized the Customer Journey} for an e-commerce platform, overseeing order lifecycles and backend integrations, by mapping the end-to-end user flow to identify high-friction drop-off points and implement UX improvements.
\item \textbf{Conducted Market Research} and user requirement gathering sessions to inform feature prioritization and roadmap planning, synthesizing key qualitative insights into actionable user stories for the development team.
\item \textbf{Drafted Technical Whitepapers} and documentation to support the GTM rollout of new platform capabilities, translating complex technical specifications into accessible marketing collateral for non-technical decision makers.
\item \textbf{Collaborated with QA and Sales teams} to ensure product releases met high-quality standards and market expectations, facilitating knowledge transfer sessions to enable sales readiness for new feature launches.
\end{itemize}
\end{rSubsection}

\end{rSection}

%----------------------------------------------------------------------------------------
%   PROJECTS
%----------------------------------------------------------------------------------------
\begin{rSection}{Projects (GTM Focus)}

\begin{rSubsection}{Thara: AI Personal Agent Ecosystem}{}{}{}
\item[] \vspace{-0.5em}
\begin{itemize}[left=-1.5em, labelsep=.1cm, labelwidth=.2cm, itemsep=0.0em]
\item \textbf{Defined the Product Vision \& Value Proposition} for a multi-agent AI ecosystem, targeting the personal productivity market by addressing the fragmentation of task and habit tracking.
\item \textbf{Validated Technical Feasibility} by prototyping the core infrastructure (LangGraph/FastAPI), ensuring the product roadmap was grounded in realistic engineering constraints.
\end{itemize}
\end{rSubsection}

\begin{rSubsection}{Google Maps: Product Strategy Teardown}{}{}{}
\item[] \vspace{-0.5em}
\begin{itemize}[left=-1.5em, labelsep=.1cm, labelwidth=.2cm, itemsep=0.0em]
\item \textbf{Conducted Market Analysis} involving \textbf{42 user interviews}, identifying a critical gap in menu accuracy and dietary filtering that caused platform drop-off.
\item \textbf{Formulated a Strategic Roadmap} to capture a high-value market segment by proposing a centralized menu system, directly addressing verified user pain points.
\end{itemize}
\end{rSubsection}

\begin{rSubsection}{Clash of Clans: Growth Strategy Teardown}{}{}{}
\item[] \vspace{-0.5em}
\begin{itemize}[left=-1.5em, labelsep=.1cm, labelwidth=.2cm, itemsep=0.0em]
\item \textbf{Developed a Growth Strategy} to drive \textbf{5-10\% organic user acquisition} by proposing a native social sharing feature, validated by quantitative player behavior analysis.
\item \textbf{Projected a 15-20\% increase in conversion} (join rates) by redesigning the Clan Discovery engine to improve user onboarding and community discovery.
\end{itemize}
\end{rSubsection}

\end{rSection}


%----------------------------------------------------------------------------------------
%   SKILLS
%----------------------------------------------------------------------------------------
\begin{rSection}{Skills}
\item[] \vspace{-0.5em}
\begin{itemize}[left=-1.5em, labelsep=.1cm, labelwidth=.2cm, itemsep=0.0em]
    \item \textbf{GTM \& Strategy:} Go-to-Market Strategy, Product Positioning, User Acquisition, Community-Led Growth, Competitive Analysis.
    \item \textbf{Growth Engineering:} n8n (Workflow Automation), Python (Scripting), SQL, Zapier, API Integration, Cursor IDE.
    \item \textbf{Content \& Operations:} Technical Writing, Notion (SOPs), Figma (Assets), Brand Strategy, Event Management.
    \item \textbf{Analytics \& Research:} Market Research, User Interviews, NPS Analysis, A/B Testing, Google Analytics.
\end{itemize}
\end{rSection}

\begin{rSection}{Certifications}
\item[] \vspace{-1em}
\begin{itemize}[left=-1.5em, labelsep=.1cm, labelwidth=.2cm, itemsep=0.0em]
\item Certified Scrum Product Owner (CSPO)
\item AWS Certified Cloud Practitioner (In Progress)
\item Google: Data Analytics Specialization
\end{itemize}
\end{rSection}

\end{document}
